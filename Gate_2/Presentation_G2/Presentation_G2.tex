\documentclass{beamer}
\usepackage{listings}
\lstset{
%language=C,
frame=single, 
breaklines=true,
columns=fullflexible
}
\usepackage{blkarray}
\usepackage{subcaption}
\usepackage{url}
\usepackage{tikz}
\usepackage{tkz-euclide} % loads  TikZ and tkz-base
%\usetkzobj{all}
\usetikzlibrary{calc,math}
\usepackage{float}
\newcommand{\myvec}[1]{\ensuremath{\begin{pmatrix}#1\end{pmatrix}}}
\newcommand{\mydet}[1]{\ensuremath{\begin{vmatrix}#1\end{vmatrix}}}
\providecommand{\brak}[1]{\ensuremath{\left(#1\right)}}
\providecommand{\sbrak}[1]{\ensuremath{{}\left[#1\right]}}
\providecommand{\lsbrak}[1]{\ensuremath{{}\left[#1\right.}}
\providecommand{\rsbrak}[1]{\ensuremath{{}\left.#1\right]}}
\newcommand\norm[1]{\left\lVert#1\right\rVert}
\renewcommand{\vec}[1]{\mathbf{#1}}
\usepackage[export]{adjustbox}
\usepackage[utf8]{inputenc}
\usepackage{amsmath}
\usepackage{physics}
\usepackage{tikz}
\usetikzlibrary{automata, positioning}
\usetheme{Boadilla}
\providecommand{\pr}[1]{\ensuremath{\Pr\left(#1\right)}}

\title{GATE 2 Presentation}
\author{Akyam L Dhatri Nanda}
\date{AI20BTECH11002}
\begin{document}
\begin{frame}
\titlepage
\end{frame}

\begin{frame}
\frametitle{Question}
\begin{block}{GATE 2010 Q.14}

Consider the Z-transform $X(z) = 5z^2 + 4z^{-1} + 3,$

$ 0 < |z| < \infty$. The inverse Z-transform x[n] is

\begin{enumerate}[label={\Alph*)}]
\item $5\delta[n+2] + 3\delta[n] + 4\delta[n-1]$
\item $5\delta[n-2] + 3\delta[n] + 4\delta[n+1]$
\item $5u[n+2] + 3u[n] + 4u[n-1]$
\item $5u[n-2] + 3u[n] + 4u[n+1]$
\end{enumerate}
\end{block}
\end{frame}

\begin{frame}{}
    \begin{block}{Theorem}
    The inverse Z-transform of X(z) is defined as 
\begin{align}
    x[n] = \frac{1}{2\pi j} \oint_{c} X(z)z^{n-1}dz \label{eq 1}
\end{align}
where c is a counter clockwise contour in the ROC of X(z) encircling the origin. 

$\implies x[n]$ = $\sum$[residues of $X(z)z^{n-1}$ at the poles inside c]

The residue at $z = d_0$ is defined as 
\begin{align}
    \frac{1}{(s-1)!}\sbrak{\frac{d^{s-1}\psi(z)}{dz^{s-1}}}_{z=d_0}
\end{align}
where
\begin{align}
X(z)Z^{n-1} = \frac{\psi(z)}{(z-d_0)^s}
\end{align}
    \end{block}
\end{frame}

\begin{frame}{Solution}
    Given, Z-transform 
\begin{align}
    X(z) = 5z^2 + 4z^{-1} + 3 \label{eq 0}
\end{align}
ROC = $0<|z|<\infty$

Now,
\begin{align}
X(z)z^{n-1} = \frac{5z^{n+2} + 3z^n + 4z^{n-1}}{z}\\
\implies \psi(z) = 5z^{n+2} + 3z^n + 4z^{n-1},\\ 
d_0 = 0, s = 1
\end{align}
\end{frame}

\begin{frame}{}
From \eqref{eq 1}
    \begin{align}
  x[n] = \frac{1}{0!}\sbrak{5z^{n+2} + 3z^n + 4z^{n-1}}_{z=0}
\end{align}

\begin{align}
    n &= -2 \implies x[-2] = 5\\ 
    n &= 0 \implies x[0] = 3\\
    n &= 1 \implies x[1] = 4
\end{align}
Therefore,
\begin{align}
    x[n] &= 
 \begin{cases}
 5, & n = -2\\
 3, & n = 0\\
 4, &  n = 1\\
 0, & \text{Otherwise}
 \end{cases}
\end{align} \label{eq 12}
\end{frame}

\begin{frame}{}
    We know that x[n] and the impulse sequence $\delta[n]$ are related by,
\begin{align}
    x[n] = \sum_{k=-\infty}^{\infty} x[k]\delta[n-k]
\end{align}
Therefore,
\begin{align}
    x[n] = 5\delta[n+2] + 3\delta[n] + 4\delta[n-1]
\end{align}

\end{frame}

\end{document}
