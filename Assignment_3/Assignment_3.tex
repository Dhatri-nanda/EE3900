\documentclass[journal,12pt,twocolumn]{IEEEtran}

\usepackage{setspace}
\usepackage{gensymb}

\singlespacing


\usepackage[cmex10]{amsmath}

\usepackage{amsthm}

\usepackage{mathrsfs}
\usepackage{txfonts}
\usepackage{stfloats}
\usepackage{bm}
\usepackage{cite}
\usepackage{cases}
\usepackage{subfig}

\usepackage{longtable}
\usepackage{multirow}

\usepackage{enumitem}
\usepackage{mathtools}
\usepackage{steinmetz}
\usepackage{tikz}
\usepackage{circuitikz}
\usepackage{verbatim}
\usepackage{tfrupee}
\usepackage[breaklinks=true]{hyperref}
\usepackage{graphicx}
\usepackage{tkz-euclide}
\usepackage{float}

\usetikzlibrary{calc,math}
\usepackage{listings}
    \usepackage{color}                                            %%
    \usepackage{array}                                            %%
    \usepackage{longtable}                                        %%
    \usepackage{calc}                                             %%
    \usepackage{multirow}                                         %%
    \usepackage{hhline}                                           %%
    \usepackage{ifthen}                                           %%
    \usepackage{lscape}     
\usepackage{multicol}
\usepackage{chngcntr}

\DeclareMathOperator*{\Res}{Res}

\renewcommand\thesection{\arabic{section}}
\renewcommand\thesubsection{\thesection.\arabic{subsection}}
\renewcommand\thesubsubsection{\thesubsection.\arabic{subsubsection}}

\renewcommand\thesectiondis{\arabic{section}}
\renewcommand\thesubsectiondis{\thesectiondis.\arabic{subsection}}
\renewcommand\thesubsubsectiondis{\thesubsectiondis.\arabic{subsubsection}}


\hyphenation{op-tical net-works semi-conduc-tor}
\def\inputGnumericTable{}                                 %%

\lstset{
%language=C,
frame=single, 
breaklines=true,
columns=fullflexible
}
\begin{document}
\newtheorem{theorem}{Theorem}[section]
\newtheorem{problem}{Problem}
\newtheorem{proposition}{Proposition}[section]
\newtheorem{lemma}{Lemma}[section]
\newtheorem{corollary}[theorem]{Corollary}
\newtheorem{example}{Example}[section]
\newtheorem{definition}[problem]{Definition}

\newcommand{\BEQA}{\begin{eqnarray}}
\newcommand{\EEQA}{\end{eqnarray}}
\newcommand{\define}{\stackrel{\triangle}{=}}
\bibliographystyle{IEEEtran}
\providecommand{\mbf}{\mathbf}
\providecommand{\pr}[1]{\ensuremath{\Pr\left(#1\right)}}
\providecommand{\qfunc}[1]{\ensuremath{Q\left(#1\right)}}
\providecommand{\sbrak}[1]{\ensuremath{{}\left[#1\right]}}
\providecommand{\lsbrak}[1]{\ensuremath{{}\left[#1\right.}}
\providecommand{\rsbrak}[1]{\ensuremath{{}\left.#1\right]}}
\providecommand{\brak}[1]{\ensuremath{\left(#1\right)}}
\providecommand{\lbrak}[1]{\ensuremath{\left(#1\right.}}
\providecommand{\rbrak}[1]{\ensuremath{\left.#1\right)}}
\providecommand{\cbrak}[1]{\ensuremath{\left\{#1\right\}}}
\providecommand{\lcbrak}[1]{\ensuremath{\left\{#1\right.}}
\providecommand{\rcbrak}[1]{\ensuremath{\left.#1\right\}}}
\theoremstyle{remark}
\newtheorem{rem}{Remark}
\newcommand{\sgn}{\mathop{\mathrm{sgn}}}
\providecommand{\abs}[1]{\vert#1\vert}
\providecommand{\res}[1]{\Res\displaylimits_{#1}} 
\providecommand{\norm}[1]{\lVert#1\rVert}
%\providecommand{\norm}[1]{\lVert#1\rVert}
\providecommand{\mtx}[1]{\mathbf{#1}}
\providecommand{\mean}[1]{E[ #1 ]}
\providecommand{\fourier}{\overset{\mathcal{F}}{ \rightleftharpoons}}
%\providecommand{\hilbert}{\overset{\mathcal{H}}{ \rightleftharpoons}}
\providecommand{\system}{\overset{\mathcal{H}}{ \longleftrightarrow}}
	%\newcommand{\solution}[2]{\textbf{Solution:}{#1}}
\newcommand{\solution}{\noindent \textbf{Solution: }}
\newcommand{\cosec}{\,\text{cosec}\,}
\providecommand{\dec}[2]{\ensuremath{\overset{#1}{\underset{#2}{\gtrless}}}}
\newcommand{\myvec}[1]{\ensuremath{\begin{pmatrix}#1\end{pmatrix}}}
\newcommand{\mydet}[1]{\ensuremath{\begin{vmatrix}#1\end{vmatrix}}}
\numberwithin{equation}{subsection}
\makeatletter
\@addtoreset{figure}{problem}
\makeatother
\let\StandardTheFigure\thefigure
\let\vec\mathbf
\renewcommand{\thefigure}{\theproblem}
\def\putbox#1#2#3{\makebox[0in][l]{\makebox[#1][l]{}\raisebox{\baselineskip}[0in][0in]{\raisebox{#2}[0in][0in]{#3}}}}
     \def\rightbox#1{\makebox[0in][r]{#1}}
     \def\centbox#1{\makebox[0in]{#1}}
     \def\topbox#1{\raisebox{-\baselineskip}[0in][0in]{#1}}
     \def\midbox#1{\raisebox{-0.5\baselineskip}[0in][0in]{#1}}
\vspace{3cm}
\title{ASSIGNMENT 3}
\author{Dhatri Nanda \\ AI20BTECH11002}
\maketitle
\newpage
\bigskip
\renewcommand{\thefigure}{\theenumi}
\renewcommand{\thetable}{\theenumi}
Download all python codes from 
\begin{lstlisting}
https://github.com/Dhatri-nanda/EE3900/blob/main/Assignment_3/code.py
\end{lstlisting}
%
and latex-tikz codes from 
%
\begin{lstlisting}
https://github.com/Dhatri-nanda/EE3900/blob/main/Assignment_3/Assignment_3.tex
\end{lstlisting}
\section{Construction 2.9}
Can you construct the quadrilateral PLAN if $PL = 6, LA = 9.5, \angle P = 75, \angle L = 150$ and $ \angle A = 140$ 
%
\section{SOLUTION}
\begin{lemma}
Given 
\begin{align}
    &PL = 6\\
    &LA = 4.5\\
    &\angle P = 75 \degree\\
    &\angle L = 150 \degree\\
    &\angle A = 140 \degree
\end{align}
Let
\begin{align}
        &\angle P=\theta \label{eq1}
    \\
    &\angle L=\alpha
    \\
    &\angle A=\delta \label{eq2}
    \\
    &\norm{\vec{L}-\vec{P}} =a, \label{eq3}
    \\
    &\norm{\vec{A}-\vec{L}} =b,\label{eq4}
    \\
 &\norm{\vec{A}-\vec{N}}=c
 \\
  &\norm{\vec{P}-\vec{N}} =d 
  \\
    &\norm{\vec{P}-\vec{A}} =e
  \\
  &\theta= \theta_1 + \theta_2
  \\
  & \delta_1=\angle NAP \\
  & \delta_2= \angle LAP\\
  & \gamma= \angle N
\end{align}
If three angles and two sides of a quadrilateral are known, then the coordinates of the vertices can be expressed as
\begin{align}
    \vec{A}=\vec{L}+b\times \myvec{\cos {(180\degree-\alpha)} \\ \sin{(180\degree - \alpha)}}\\
    \vec{N}=d \myvec{\cos \theta \\ \sin \theta}
\end{align}
Where
    \begin{align}
    &d=e \times \brak{\frac{\sin\brak{{\delta-\sin^{-1}\sbrak{\sin{\alpha}\times \brak{\frac{b}{e}}}}}}{\sin\brak{{360\degree-(\alpha+\theta+\delta)}}}}\\
   & e=\sqrt{a^2+b^2-2\times a \times b\cos{\alpha}}
\end{align}
\end{lemma}
\begin{proof}
Using angle sum rule of quadrilaterals
\begin{align}
\gamma=360\degree-(\alpha+\theta+\delta)
\end{align}
 Now, using cosine formula in $\triangle PLA$ we can find e:
\begin{align}
 e^2=
a^2+b^2-2\times a \times b\cos{\alpha}
\end{align}
Using sine rule,
\begin{align}
    \frac{\sin{\alpha}}{e}=\frac{\sin{\delta_2}}{b}\\
    \delta_2=\sin^{-1}\sbrak{\sin{\alpha}\times \brak{\frac{b}{e}}}\\
\end{align}
  Now in $\triangle MER$,
  \begin{align}
  \delta_1=\delta-\delta_2
  \end{align}
Using sine law of triangle,
\begin{align}
    \frac{\sin{\gamma}}{e}&=\frac{\sin{\delta_1}}{d}\\
    \implies d&=e\times \brak{\frac{\sin{\delta_1}}{\sin{\gamma}}}
\end{align}
From the above equations, we get
\begin{align}
d=e \times \brak{\frac{\sin\brak{{\delta-\sin^{-1}\sbrak{\sin{\alpha}\times \brak{\frac{b}{e}}}}}}{\sin\brak{{360\degree-(\alpha+\theta+\delta)}}}}
\end{align}
where
\begin{align}
    e=\sqrt{a^2+b^2-2\times a \times b\cos{\alpha}}
\end{align}
\end{proof}
   
Calculating e 
\begin{align}
    e &= \sqrt{6^2 + 9.5^2 -2\times 6.5\times 9\times\cos{150}}\\
    &\approx 15
\end{align}
 Calculating d
 
 \begin{align}
     d &= 15 \times \brak{\frac{\sin\brak{{140-\sin^{-1}\sbrak{\sin{150}\times \brak{\frac{9.5}{15}}}}}}{\sin\brak{{360\degree-(75+150+140)}}}}
 \end{align}
    Here, the denominator is $\sin\brak{{-5}}$, which is a negative value
    
    As d is a side of the quadrilateral, it cannot be negative, so we cannot construct the quadrilateral PLAN.
\end{document}