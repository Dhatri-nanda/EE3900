\documentclass[journal,12pt,twocolumn]{IEEEtran}

\usepackage{graphicx}
\usepackage{setspace}
\usepackage{gensymb}
\singlespacing
\usepackage[cmex10]{amsmath}
\usepackage{amssymb}
\usepackage{xurl}
\usepackage{tabularx}
\usepackage{amsthm}
\usepackage{comment}
\usepackage{mathrsfs}
\usepackage{txfonts}
\usepackage{stfloats}
\usepackage{bm}
\usepackage{cite}
\usepackage{cases}
\usepackage{subfig}
\usepackage{arydshln}
\usepackage{longtable}
\usepackage{multirow}

\usepackage{enumitem}
\usepackage{mathtools}
\usepackage{steinmetz}
\usepackage{tikz}
\usepackage{circuitikz}
\usepackage{verbatim}
\usepackage{tfrupee}
\usepackage[breaklinks=true]{hyperref}
\usepackage{graphicx}
\usepackage{tkz-euclide}
\usetikzlibrary{automata, positioning}
\usetikzlibrary{calc,math}
\usepackage{listings}
    \usepackage{color}                                            %%
    \usepackage{array}                                            %%
    \usepackage{longtable}                                        %%
    \usepackage{calc}                                             %%
    \usepackage{multirow}                                         %%
    \usepackage{hhline}                                           %%
    \usepackage{ifthen}                                           %%
    \usepackage{lscape}     
\usepackage{multicol}
\usepackage{chngcntr}
\usepackage{blkarray}

\DeclareMathOperator*{\Res}{Res}

\renewcommand\thesection{\arabic{section}}
\renewcommand\thesubsection{\thesection.\arabic{subsection}}
\renewcommand\thesubsubsection{\thesubsection.\arabic{subsubsection}}

\renewcommand\thesectiondis{\arabic{section}}
\renewcommand\thesubsectiondis{\thesectiondis.\arabic{subsection}}
\renewcommand\thesubsubsectiondis{\thesubsectiondis.\arabic{subsubsection}}


\hyphenation{op-tical net-works semi-conduc-tor}
\def\inputGnumericTable{}                                 %%

\lstset{
%language=C,
frame=single, 
breaklines=true,
columns=fullflexible
}
\begin{document}


\newtheorem{theorem}{Theorem}[section]
\newtheorem{problem}{Problem}
\newtheorem{proposition}{Proposition}[section]
\newtheorem{lemma}{Lemma}[section]
\newtheorem{corollary}[theorem]{Corollary}
\newtheorem{example}{Example}[section]
\newtheorem{definition}[problem]{Definition}

\newcommand{\BEQA}{\begin{eqnarray}}
\newcommand{\EEQA}{\end{eqnarray}}
\newcommand{\define}{\stackrel{\triangle}{=}}
\bibliographystyle{IEEEtran}
\raggedbottom
\setlength{\parindent}{0pt}
\providecommand{\mbf}{\mathbf}
\providecommand{\pr}[1]{\ensuremath{\Pr\left(#1\right)}}
\providecommand{\qfunc}[1]{\ensuremath{Q\left(#1\right)}}
\providecommand{\sbrak}[1]{\ensuremath{{}\left[#1\right]}}
\providecommand{\lsbrak}[1]{\ensuremath{{}\left[#1\right.}}
\providecommand{\rsbrak}[1]{\ensuremath{{}\left.#1\right]}}
\providecommand{\brak}[1]{\ensuremath{\left(#1\right)}}
\providecommand{\lbrak}[1]{\ensuremath{\left(#1\right.}}
\providecommand{\rbrak}[1]{\ensuremath{\left.#1\right)}}
\providecommand{\cbrak}[1]{\ensuremath{\left\{#1\right\}}}
\providecommand{\lcbrak}[1]{\ensuremath{\left\{#1\right.}}
\providecommand{\rcbrak}[1]{\ensuremath{\left.#1\right\}}}
\theoremstyle{remark}
\newtheorem{rem}{Remark}
\newcommand{\sgn}{\mathop{\mathrm{sgn}}}
\providecommand{\abs}[1]{\vert#1\vert}
\providecommand{\res}[1]{\Res\displaylimits_{#1}} 
\providecommand{\norm}[1]{\lVert#1\rVert}
%\providecommand{\norm}[1]{\lVert#1\rVert}
\providecommand{\mtx}[1]{\mathbf{#1}}
\providecommand{\mean}[1]{E[ #1 ]}
\providecommand{\fourier}{\overset{\mathcal{F}}{ \rightleftharpoons}}
%\providecommand{\hilbert}{\overset{\mathcal{H}}{ \rightleftharpoons}}
\providecommand{\system}{\overset{\mathcal{H}}{ \longleftrightarrow}}
	%\newcommand{\solution}[2]{\textbf{Solution:}{#1}}
\newcommand{\solution}{\noindent \textbf{Solution: }}
\newcommand{\cosec}{\,\text{cosec}\,}
\providecommand{\dec}[2]{\ensuremath{\overset{#1}{\underset{#2}{\gtrless}}}}
\newcommand{\myvec}[1]{\ensuremath{\begin{pmatrix}#1\end{pmatrix}}}
\newcommand{\mydet}[1]{\ensuremath{\begin{vmatrix}#1\end{vmatrix}}}
\newcommand*{\permcomb}[4][0mu]{{{}^{#3}\mkern#1#2_{#4}}}
\newcommand*{\perm}[1][-3mu]{\permcomb[#1]{P}}
\newcommand*{\comb}[1][-1mu]{\permcomb[#1]{C}}
\numberwithin{equation}{subsection}
\makeatletter
\@addtoreset{figure}{problem}
\makeatother
\let\StandardTheFigure\thefigure
\let\vec\mathbf
\renewcommand{\thefigure}{\theproblem}
\def\putbox#1#2#3{\makebox[0in][l]{\makebox[#1][l]{}\raisebox{\baselineskip}[0in][0in]{\raisebox{#2}[0in][0in]{#3}}}}
     \def\rightbox#1{\makebox[0in][r]{#1}}
     \def\centbox#1{\makebox[0in]{#1}}
     \def\topbox#1{\raisebox{-\baselineskip}[0in][0in]{#1}}
     \def\midbox#1{\raisebox{-0.5\baselineskip}[0in][0in]{#1}}
\vspace{3cm}
\title{\textbf{ GATE ASSIGNMENT 4}}
\author{Dhatri Nanda - AI20BTECH11002}
\maketitle
\newpage
\bigskip
\renewcommand{\thefigure}{\arabic{figure}}
\renewcommand{\thetable}{\arabic{table}}
Download latex-tikz codes from 
%
\begin{lstlisting}
https://github.com/Dhatri-nanda/EE3900/blob/main/Gate_4/Gate_4.tex
\end{lstlisting}
\section*{QUESTION}
The Fourier Transform of the signal $f(t) = e^{-3t^2}$ is of the following form, where A and B are constants

\begin{enumerate}[label={\Alph*)}]
\item $Ae^{-B|f|}$\\
\item $Ae^{-Bf^2}$\\
\item $A + B|f|^2$\\
\item $Ae^{-Bf}$\\
\end{enumerate}

\section*{SOLUTION}
        Given, 
        \begin{align}
            f(t) = e^{-at^2} \label{eq 1}
        \end{align}
        
        The fourier transform of a given signal f(t) is given by,
        \begin{align}
            X(\omega) = \int_{-\infty}^{\infty} f(t) e^{-j\omega t} dt
        \end{align}
       So let,
        \begin{align}
            f(t) \fourier F(\omega)
        \end{align}
        The fourier transform of $\frac{d f(t)}{dt}$ is given by,
        \begin{align}
            &= \int_{-\infty}^{\infty}  \frac{df(t)}{dt} e^{-j\omega t} dt\\
            &= \sbrak{e^{-j\omega t} f(t)}_{-\infty}^{\infty} + j\omega \int_{-\infty}^{\infty} f(t) e^{-j\omega t} dt\\
            &= j\omega F(\omega)
        \end{align}
        Therefore,
        \begin{align}
            \frac{df(t)}{dt} \fourier j\omega F(\omega) \label {eq 0.0.6}
        \end{align}
        
        From \eqref{eq 0.0.6} and \eqref{eq 1}
        \begin{align}
            te^{-at^2} \fourier \frac{-j\omega}{2a} F(\omega) \label{eq 0.0.7}
        \end{align}
        
        \begin{lemma}
        $te^{-at^2} \fourier j\frac{dF(\omega)}{d\omega}$ 
        \label{lemma 1}
        \end{lemma}
        \begin{proof}
        We prove it by finding the inverse fourier transform of $j\frac{dF(\omega)}{d\omega}$
        
        The inverse fourier transform is given by
        \begin{align}
            f^1(t) &= \frac{1}{2\pi} \int_{-\infty}^{\infty} F^1(\omega)e^{-j\omega t} d\omega\\\label{eq 0.0.8}
            &= \frac{1}{2\pi} \int_{-\infty}^{\infty} j\frac{dF(\omega)}{d\omega}
            e^{j\omega t} d\omega\\
            &= \frac{1}{2\pi} \sbrak{je^{j\omega t} F(\omega)}_{-\infty}^{\infty} + \int_{-\infty}^{\infty} te^{j\omega t} F(\omega) d\omega \\
            &= \frac{1}{2\pi} \int_{-\infty}^{\infty} t F(\omega) e^{j\omega t} d\omega 
        \end{align}
        From \eqref{eq 0.0.8}, 
        \begin{align}
            f^1(t) &= tf(t)\\
            &= te^{-at^2} 
        \end{align}
        \end{proof}
        
        From \eqref{eq 0.0.6} and Lemma \ref{lemma 1} 
        \begin{align}
            j\frac{dF(\omega)}{d\omega} &= \frac{-j\omega}{2a} F(\omega)\\
           \implies \frac{dF(\omega)}{F(\omega)} &= \frac{-1}{2a} \omega d\omega
        \end{align}
        
        On integrating,
        \begin{align}
            log(F(\omega)) = \frac{-1}{2a}.\frac{\omega^2}{2} = -\frac{\omega^2}{4a} \\
            \implies F(\omega) = e^{-\frac{\omega^2}{4a}} 
        \end{align}
        
        Putting $\omega = 2\pi f$ and $a = 3$
        \begin{align}
            F(\omega) = e^\frac{-\pi^2 f^2}{3}
        \end{align}
        
       $\therefore$ correct option is B
\end{document}
